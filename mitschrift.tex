\documentclass[12pt,DIV=15,a4paper,twoside,draft=false]{scrartcl}
\usepackage[utf8]{inputenc}
\usepackage[ngerman]{babel}
\usepackage{makeidx}
\usepackage{graphicx}
\usepackage{dsfont}
\usepackage{ulem}
\usepackage{amsthm}
\usepackage{amsmath}
\usepackage{wrapfig}
\usepackage{pdfpages}
\usepackage{uniinput}
\usepackage{float}

% Floats
\floatstyle{ruled}
\newfloat{bemer}{H}{lomark}
\floatname{bemer}{Bemerkung}

% Formatiert die Kopfzeilen und Fusszeilen
\usepackage{fancyhdr}
\pagestyle{fancy}
\setlength{\headheight}{25.2pt} % Damit alles in den Header passt.
\fancyhead[ER,OL]{\begin{minipage}{\textwidth}\begin{flushright}
\leftmark \\ \today
\end{flushright}
\end{minipage}}
\fancyhead[EL,OR]{\begin{minipage}{\textwidth}Prof. Ratzinger\\ \rightmark
\end{minipage}}
\fancyfoot[EL,OR]{}
%\setlength{\headheight}{25.2pt} % Damit alles in den Header passt.
%\fancyhead[R]{\begin{minipage}{\textwidth}
%\leftmark \\ \today
%\end{minipage}}
%\fancyhead[L]{\begin{minipage}{\textwidth}Prof. Ratzinger\\ \rightmark \end{minipage}}

\newcommand{\ket}[1]{\left\vert #1 \right\rangle}
\newcommand{\bra}[1]{\left\langle #1 \right\vert}
\newcommand{\braket}[2]{\langle #1 \vert #2 \rangle}
\newcommand{\Braket}[3]{\langle #1 \vert #2 \vert #3 \rangle}
\newcommand{\ketbra}[2]{\vert #1 \rangle \langle #2 \vert}
\newcommand{\kommut}[2]{[\hat{#1}, \hat{#2}]}
\newcommand{\antikommut}[2]{\{\hat{#1}, \hat{#2}\}}


% Titelseite
\author{Jonathan Pieper}
\title{Einführung in die Beschleunigerphysik}
\subtitle{Mitschrift zur Vorlesung von Prof. Ratzinger}

\usepackage[colorlinks=true,linkcolor=black]{hyperref}

\begin{document}
\begin{titlepage}
\maketitle
\tableofcontents
\end{titlepage}

\section{Vorbesprechung}
\paragraph{Übung:} 8:45 -- 9:30 Uhr
\paragraph{Vorlesung:} 9:45 -- 11:15 Uhr


\section[Einführung]{Einführung - Wozu dienen Teilchenstrahlen?}
\subsection{Teilchenstrahlen in der Grundlagenforschung}
Aktuelle Fragen:
\begin{itemize}
	\item Wie werden Quarks und Gluonen frei (deconfinement)?
	\item Wie entsteht die Hadronenmasse sowie der Hadronenspin aus den Konstituenten?
	\item Warum haben wir einen Überschuss an Materie gegenüber Antimaterie (das was wir heute sehen)?
\end{itemize}


\subsection{Teilchenstrahlung in der Angewandten Forschung}
Synchrotronstrahlungsquellen, 
Free Electron Laser(FEL)

\paragraph{Energieversorgung}
\begin{itemize}
\item  \textbf{Transmutation} von radioaktivem Abfall aus Spaltungsreaktionen (MYRRHA, Belgien)

\item Teststrahlen für die deuterium Fusionsforschung (\textbf{IFMIF}, 250\;{}mA Deuteronen auf Lithium erzeugen eine intensive 13\;{}MeV Neutronenstrahlung)

\item \textbf{Trägheitsfusion} mittels Schwerionentreiberstrahl (zur Zeit nicht realistisch aufgrund kurzer Strahllebensdauern)
\end{itemize}
$$ \mathrm{d + t \rightarrow {}^4 He + r_i + 17\;{}MeV} $$

\subparagraph{Medizin}
Produktion radioaktiver Isotope als \textbf{Tracer} ($\mathrm{Tc^{99}}$ als $γ$ - Emitter nach Andocken an ein interessierendes Molekül)

Positronen-Emissions-Tomographie \textbf{PET} mit kurzlebigen Isotopen wie $\mathrm{F^{18}}$, \textbf{innere Radionuklidtherapie}

\textbf{Krebstherapie} mittels Elektronen-, Protonen- und leichten Ionenstrahlen\\
Vorteil des Bragg-Peaks bei Protonen und leichten Ionenstrahlen (bis hinauf zum Kohlenstoff)

\paragraph{Industrie}
\textbf{Röntgenstrahlung} (2-dim Projektion, Litographie (Chip - Miniaturisierung))

\textbf{Lebensmittelbehandlung} (Sterilisierung)

\textbf{Ionenmanipulation} (Dotierung von Halbleitern)

\end{document}