\documentclass[12pt,DIV=15,a4paper,twoside,draft=false]{scrartcl}
\usepackage[utf8]{inputenc}
\usepackage[ngerman]{babel}
\usepackage{makeidx}
\usepackage{graphicx}
\usepackage{dsfont}
\usepackage{ulem}
\usepackage{amsthm}
\usepackage{amsmath}
\usepackage{wrapfig}
\usepackage{pdfpages}
\usepackage{uniinput}
\usepackage{float}

% Floats
\floatstyle{ruled}
\newfloat{bemer}{H}{lomark}
\floatname{bemer}{Bemerkung}

% Formatiert die Kopfzeilen und Fusszeilen
\usepackage{fancyhdr}
\pagestyle{fancy}
\setlength{\headheight}{25.2pt} % Damit alles in den Header passt.
\fancyhead[ER,OL]{\begin{minipage}{\textwidth}\begin{flushright}
\leftmark \\ \today
\end{flushright}
\end{minipage}}
\fancyhead[EL,OR]{\begin{minipage}{\textwidth}Prof. Ratzinger\\ \rightmark
\end{minipage}}
\fancyfoot[EL,OR]{}
%\setlength{\headheight}{25.2pt} % Damit alles in den Header passt.
%\fancyhead[R]{\begin{minipage}{\textwidth}
%\leftmark \\ \today
%\end{minipage}}
%\fancyhead[L]{\begin{minipage}{\textwidth}Prof. Ratzinger\\ \rightmark \end{minipage}}

\newcommand{\ket}[1]{\left\vert #1 \right\rangle}
\newcommand{\bra}[1]{\left\langle #1 \right\vert}
\newcommand{\braket}[2]{\langle #1 \vert #2 \rangle}
\newcommand{\Braket}[3]{\langle #1 \vert #2 \vert #3 \rangle}
\newcommand{\ketbra}[2]{\vert #1 \rangle \langle #2 \vert}
\newcommand{\kommut}[2]{[\hat{#1}, \hat{#2}]}
\newcommand{\antikommut}[2]{\{\hat{#1}, \hat{#2}\}}


% Titelseite
\author{Jonathan Pieper}
\title{Einführung in die Beschleunigerphysik}
\subtitle{Mitschrift zur Vorlesung von Prof. Ratzinger}

\usepackage[colorlinks=true,linkcolor=black]{hyperref}

\begin{document}
\begin{titlepage}
\maketitle
\tableofcontents
\end{titlepage}

\subsection{Vorbesprechung}
\paragraph{Übung:} 8:45 -- 9:30 Uhr
\paragraph{Vorlesung:} 9:45 -- 11:15 Uhr


\section[Einführung]{Einführung - Wozu dienen Teilchenstrahlen?}
\subsection{Teilchenstrahlen in der Grundlagenforschung}
Aktuelle Fragen:
\begin{itemize}
	\item Wie werden Quarks und Gluonen frei (deconfinement)?
	\item Wie entsteht die Hadronenmasse sowie der Hadronenspin aus den Konstituenten?
	\item Warum haben wir einen Überschuss an Materie gegenüber Antimaterie (das was wir heute sehen)?
\end{itemize}


\subsection{Teilchenstrahlung in der Angewandten Forschung}
Synchrotronstrahlungsquellen, 
Free Electron Laser(FEL)

\paragraph{Energieversorgung}
\begin{itemize}
\item  \textbf{Transmutation} von radioaktivem Abfall aus Spaltungsreaktionen (MYRRHA, Belgien)

\item Teststrahlen für die deuterium Fusionsforschung (\textbf{IFMIF}, 250\;{}mA Deuteronen auf Lithium erzeugen eine intensive 13\;{}MeV Neutronenstrahlung)

\item \textbf{Trägheitsfusion} mittels Schwerionentreiberstrahl (zur Zeit nicht realistisch aufgrund kurzer Strahllebensdauern)
\end{itemize}
$$ \mathrm{d + t \rightarrow {}^4 He + r_i + 17\;{}MeV} $$

\subparagraph{Medizin}
Produktion radioaktiver Isotope als \textbf{Tracer} ($\mathrm{Tc^{99}}$ als $γ$ - Emitter nach Andocken an ein interessierendes Molekül)

Positronen-Emissions-Tomographie \textbf{PET} mit kurzlebigen Isotopen wie $\mathrm{F^{18}}$, \textbf{innere Radionuklidtherapie}

\textbf{Krebstherapie} mittels Elektronen-, Protonen- und leichten Ionenstrahlen\\
Vorteil des Bragg-Peaks bei Protonen und leichten Ionenstrahlen (bis hinauf zum Kohlenstoff)

\paragraph{Industrie}
\textbf{Röntgenstrahlung} (2-dim Projektion, Litographie (Chip - Miniaturisierung))

\textbf{Lebensmittelbehandlung} (Sterilisierung)

\textbf{Ionenmanipulation} (Dotierung von Halbleitern)


\subsection[Stoßkinematik]{Stoßkinematik, Collider- und „Fixed-Target“ -- Experimente}
\paragraph{Beschreibung relativistischer Teilchenbewegungen}
\begin{align*}
β &= \frac{v}{c} & γ &= \frac{1}{\sqrt{(1-β)^2}}\\
E_0 &= mc^2 & E &= E_0 + E_{kin} = γmc^2\\
p &= βγmc
\end{align*}
Viererimpuls eines Teilchens, bzw. eines Teilchenensembles:
\begin{align*}
\begin{pmatrix}
\frac{E}{c} \\ p_x\\ p_y\\ p_z
\end{pmatrix}
 && 
\begin{pmatrix}
Σ_i \frac{E_i}{c} \\ Σ_i p_{x,i}\\ Σ_i p_{y,i}\\ Σ_i p_{z,i}
\end{pmatrix}
\end{align*}

\subsection{Wirkende elektromagnetische Feldkräfte}
$$ \vec{F} = q (\vec{E} + \vec{v}\times \vec{B}) $$
Longitudinale Impulsänderung:
$$ qE_{long} = \dot{p}_{long} = \dot{γ} m v + γ m \dot{v}_{long} = m γ^3 \dot{v}_{long} $$
Transversale elektrische und magnetische Felder zwingen Teilchen auf Kreisbahn. Dabei gilt für den Krümmungsradius:
\begin{align*}
q E_{trans} &= \frac{γ m v^2}{R} &
q v B_{trans} &= \frac{γ m v^2}{R}
\end{align*}

\section{Beschleunigerkonzepte}
\subsection[LINAC]{Lineare Beschleuniger}
zwei Hauptkonzepte
\begin{itemize}
\item \textbf{Van de Graaff} Bandgenerator, Aufladen einer geeignet geformten Elektrode
\item \textbf{Cockroft-Walton} Hochpumpen von Ladungsportionen über einen Kaskadengenerator
\end{itemize}

\subsection{Zyklotrons}
\subsubsection{Synchrotron}
Magnetfeld und Hochfrequenz werden synchron mit der Energiezunahme des Strahls so erhöht, dass immer der gleiche Orbit durchlaufen wird.
\begin{align*}
R &= \frac{p}{qB} = \frac{β E}{c q B} = const. &
E &= γmc^2
\end{align*}

Bei gegebenem Umfang $U$ gilt für die Umlauffrequenz $f_{rev}$ und die Betriebsfroquenz $f$ auf der Harmonischen $h$:
\begin{align*}
 f_{rev}(t) &= \frac{c}{U} β(t) &
 f = h \cdot f_{rev}
\end{align*}

Synchrotrons werden durch ihren $B_{max}\cdot ρ$--Wert gekennzeichnet, welcher linear mit $p_{max}$ bzw. $E_{max}$ zusammenhängt ($ρ$: Krümmungsradius):
$$ B_{max} ρ = \frac{p_{max}}{q} = \frac{β_{max} E_{max}}{cq} = \frac{β_{max} γ_{max} mc}{q} $$


\paragraph{Injektion in ein Synchrotron}
\begin{itemize}
\item Singe Turn
\item Multi Turn
\item Non Lionvillian
\end{itemize}



\end{document}