\documentclass[12pt,DIV=15,a4paper,twoside,draft=false]{scrartcl}
\usepackage[utf8]{inputenc}
\usepackage[ngerman]{babel}
\usepackage{makeidx}
\usepackage{graphicx}
\usepackage{dsfont}
\usepackage{ulem}
\usepackage{amsthm}
\usepackage{amsmath}
\usepackage{wrapfig}
\usepackage{pdfpages}
\usepackage{uniinput}
\usepackage{float}
\usepackage{titlesec}

% Floats
\floatstyle{ruled}
\newfloat{bemer}{H}{lomark}
\floatname{bemer}{Bemerkung}

% Formatiert die Kopfzeilen und Fusszeilen
\usepackage{fancyhdr}
\pagestyle{fancy}
\setlength{\headheight}{25.2pt} % Damit alles in den Header passt.
\fancyhead[ER,OL]{\begin{minipage}{\textwidth}\begin{flushright}
\leftmark \\ \today
\end{flushright}
\end{minipage}}
\fancyhead[EL,OR]{\begin{minipage}{\textwidth}Abgabe von Jonathan Pieper \\ Übung zur Einführung in die Beschleunigerphysik
\end{minipage}}
\fancyfoot[EL,OR]{}
%\setlength{\headheight}{25.2pt} % Damit alles in den Header passt.
%\fancyhead[R]{\begin{minipage}{\textwidth}
%\leftmark \\ \today
%\end{minipage}}
%\fancyhead[L]{\begin{minipage}{\textwidth}Prof. Ratzinger\\ \rightmark \end{minipage}}

\newcommand{\antikommut}[2]{\{\hat{#1}, \hat{#2}\}}


% Titelseite
\author{Jonathan Pieper}
\title{Abgabe Übung 1}
\subtitle{Übung zur Einführung in die Beschleunigerphysik}

\usepackage[colorlinks=true,linkcolor=black]{hyperref}

\begin{document}
\maketitle
\titlelabel{Aufgabe 1.\thesection{} }

\section{Coulomb Barriere}
Die Goldfolie hat eine Dichte von $n=19.3$\;g/cm$^{3} = 19\,{}300$\;kg/m$^3$. Mit $\frac{1}{2} m_p v_0^2 = 3.5\;\mathrm{MeV}$ folgt für $v_0 = \sqrt{\frac{7\;\mathrm{MeV}}{m_p}} = 2.589\cdot 10^7$\;m/s ${}\approx 0.086 c$.

Solved by Wolframalpha:
\begin{eqnarray*}
P(θ∈[29.5^\circ , 30.5^{\circ}]) &=&
 ∫_{\frac{29.5^\circ \cdot 2 \cdot \pi}{360}}^{\frac{30.5^{\circ} \cdot 2 \cdot \pi}{360}}
\frac{Z^2 q^4 d n}{(4 \pi ε_0 m v_0^2)^2 \sin^4(θ/2)} dθ\\
&=& ∫_{\frac{29.5^\circ \cdot 2 \cdot \pi}{360}}^{\frac{30.5^{\circ} \cdot 2 \cdot \pi}{360}}
 \frac{79^2 (1.602177\cdot 10^{-19})^4 \cdot 5\cdot 10^{-6} \cdot  19300}{(4 \pi \cdot 8.854 \cdot 10^{-12}\cdot  1.672622\cdot 10^{-27} (2.589\cdot 10^7)^2)^2 sin^4(θ/2)} dθ\\ % = 9.92848 \cdot 10^-29
&=& 2.55036\cdot 10^{-29} ∫_{\frac{29.5^\circ \cdot 2 \cdot \pi}{360}}^{\frac{30.5^{\circ} \cdot 2 \cdot \pi}{360}}
\frac{dθ}{\sin^4(θ/2)}\\
& = & 2.55036 \cdot 10^{-29} \cdot 3.89298 = 9.92848 \cdot 10^{-29}
\end{eqnarray*}
Oder:
\begin{eqnarray*}
P(θ=30^{\circ}, dΘ=±0.5^{\circ}) &=&
 ∫_{-\frac{\pi}{360}}^{\frac{\pi}{360}}
\frac{Z^2 q^4 d n}{(4 \pi ε_0 m v_0^2)^2 \sin^4(30^\circ \pi /360)} dΘ\\
&=& 2.55036\cdot 10^{-29} ∫_{-\frac{\pi}{360}}^{\frac{\pi}{360}}
\frac{dΘ}{\sin^4(30^\circ \pi /360)}\\
& = & 2.55036 \cdot 10^{-29}\cdot 3.88949 = 9.92848 \cdot 10^{-29}
\end{eqnarray*}


\section{Teilchenenergien im Synchrotron}
Für den Impuls $p$ eines Teilchens im Synchrotron gilt:
$$ p = q B ρ $$
dabei ist $q$ die Ladung des Teilchens und $ρ$ der Radius der Kreisbahn. Der LHC am CERN hat einen Umfang von ${26659}$\;m. Also folgt (mit $q=e$ und $m=m_p$):
\begin{align*}
\frac{p}{q} =& 8.5\;\mathrm{T} \cdot \frac{26659\;\mathrm{m}}{2π} = 36064.75\;\mathrm{Tm} &&\textbf{magnetische Steifigkeit}\\
E =& \sqrt{(m c^2)^2 + (pc)^2} = 1.732308 \cdot 10^{-6}\;\mathrm{J} = 
10.8122\;\mathrm{TeV} && \textbf{Energie}
\end{align*}

\section{Auflösung von Teilchenstrahlen}
Für die Untersuchung des Gitters mit $γ$ Quanten muss die Wellenlänge $λ < d = 5.75\;{}$\AA{} kleiner als die Gitterkonstante sein.
Die Energie der Photonen muss demnach
\begin{eqnarray*}
λ & = & \frac{h\cdot c}{E} < 5.75\cdot 10^{-10}\;\mathrm{m} \\
\Rightarrow E  & > & \frac{1.986446\cdot 10^{-25}\;\mathrm{J m}}{5.75\cdot 10^{-15}\;\mathrm{m}} = 1.142\cdot 10^{-9}\;\mathrm{J} = 7.128\;\mathrm{GeV}
\end{eqnarray*}
Demnach wird zum erzeugen der Photonen mittels Elektronen über Bremsstrahlung eine Mindestenergie von
$$ E_{el} = e\cdot U = 7.128\cdot 10^9\;\mathrm{eV}$$
und demzufolge eine Spannung von $U = 7.128\cdot 10^9\;\mathrm{V}$ benötigt.

%\includegraphics[scale=1]{ue1-qrcode}
%\url{https://desdemonda.dyndns.org/?id=PWqPojnmpw}
\end{document}