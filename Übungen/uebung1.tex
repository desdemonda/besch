\documentclass[12pt,DIV=15,a4paper,twoside,draft=false]{scrartcl}
\usepackage[utf8]{inputenc}
\usepackage[ngerman]{babel}
\usepackage{makeidx}
\usepackage{graphicx}
\usepackage{dsfont}
\usepackage{ulem}
\usepackage{amsthm}
\usepackage{amsmath}
\usepackage{wrapfig}
\usepackage{pdfpages}
\usepackage{uniinput}
\usepackage{float}
\usepackage{titlesec}

% Floats
\floatstyle{ruled}
\newfloat{bemer}{H}{lomark}
\floatname{bemer}{Bemerkung}

% Formatiert die Kopfzeilen und Fusszeilen
\usepackage{fancyhdr}
\pagestyle{fancy}
\setlength{\headheight}{25.2pt} % Damit alles in den Header passt.
\fancyhead[ER,OL]{\begin{minipage}{\textwidth}\begin{flushright}
\leftmark \\ \today
\end{flushright}
\end{minipage}}
\fancyhead[EL,OR]{\begin{minipage}{\textwidth}Abgabe von Jonathan Pieper \\ Übung zur Einführung in die Beschleunigerphysik
\end{minipage}}
\fancyfoot[EL,OR]{}
%\setlength{\headheight}{25.2pt} % Damit alles in den Header passt.
%\fancyhead[R]{\begin{minipage}{\textwidth}
%\leftmark \\ \today
%\end{minipage}}
%\fancyhead[L]{\begin{minipage}{\textwidth}Prof. Ratzinger\\ \rightmark \end{minipage}}

\newcommand{\antikommut}[2]{\{\hat{#1}, \hat{#2}\}}


% Titelseite
\author{Jonathan Pieper}
\title{Abgabe Übung 1}
\subtitle{Übung zur Einführung in die Beschleunigerphysik}

\usepackage[colorlinks=true,linkcolor=black]{hyperref}

\begin{document}
\maketitle
\titlelabel{Aufgabe 1.\thesection{} }

\section{Coulomb Barriere}
Solved by Wolframalpha:
\begin{eqnarray*}
P(θ∈[29.5^\circ , 30.5^{\circ}]) &=& ∫_{\frac{29.5^\circ \cdot 2 \cdot \pi}{360}}^{\frac{30.5^{\circ} \cdot 2 \cdot \pi}{360}}
\frac{q^4 d n}{(4 \pi ε_0 m v_0^2)^2 \sin^4(θ/2)} dθ = (1.23263\cdot 10^-7 n q^4)/(e_0^2 m^2 v_0^4)
\end{eqnarray*}

\section{Teilchenenergien im Synchrotron}

\section{Auflösung von Teilchenstrahlen}
Für die Untersuchung des Gitters mit $γ$ Quanten muss die Wellenlänge $λ < d = 5.75\;{}$\AA{} kleiner als die Gitterkonstante sein.
Die Energie der Photonen muss demnach
\begin{eqnarray*}
λ & = & \frac{h\cdot c}{E} < 5.75\cdot 10^{-10} m \\
\Rightarrow E  & > & \frac{1.986446\cdot 10^{-25}\;\mathrm{J m}}{5.75\cdot 10^{-15} m} = 1.142\cdot 10^{-9}\;\mathrm{J} = 7.128\;\mathrm{GeV}
\end{eqnarray*}
Demnach wird zum erzeugen der Photonen mittels Elektronen über Bremsstrahlung eine Mindestenergie von
$$ E_{el} = e\cdot U = 7.128\cdot 10^9\;\mathrm{eV}$$
und demzufolge eine Spannung von $U = 7.128\cdot 10^9\;\mathrm{V}$ benötigt.
\end{document}