\documentclass[12pt,DIV=15,a4paper,twoside,draft=false]{scrartcl}
\usepackage[utf8]{inputenc}
\usepackage[ngerman]{babel}
\usepackage{makeidx}
\usepackage{graphicx}
%\usepackage{dsfont}
\usepackage{ulem}
\usepackage{amsthm}
\usepackage{amsmath}
\usepackage{wrapfig}
\usepackage{pdfpages}
\usepackage{uniinput}
\usepackage{float}
\usepackage{titlesec}

% Floats
\floatstyle{ruled}
\newfloat{bemer}{H}{lomark}
\floatname{bemer}{Bemerkung}

% Formatiert die Kopfzeilen und Fusszeilen
\usepackage{fancyhdr}
\pagestyle{fancy}
\setlength{\headheight}{25.2pt} % Damit alles in den Header passt.
\fancyhead[ER,OL]{\begin{minipage}{\textwidth}\begin{flushright}
\leftmark \\ \today
\end{flushright}
\end{minipage}}
\fancyhead[EL,OR]{\begin{minipage}{\textwidth}Abgabe von Jonathan Pieper \\ Übung zur Einführung in die Beschleunigerphysik
\end{minipage}}
\fancyfoot[EL,OR]{}
%\setlength{\headheight}{25.2pt} % Damit alles in den Header passt.
%\fancyhead[R]{\begin{minipage}{\textwidth}
%\leftmark \\ \today
%\end{minipage}}
%\fancyhead[L]{\begin{minipage}{\textwidth}Prof. Ratzinger\\ \rightmark \end{minipage}}

\newcommand{\antikommut}[2]{\{\hat{#1}, \hat{#2}\}}
\newcommand{\blatt}{3}

% Titelseite
\author{Jonathan Pieper\and Dominik Kaufhold\and Kai Harry Schmidt\and Tobin Knautz}
\title{Abgabe Übung \blatt{}}
\subtitle{Übung zur Einführung in die Beschleunigerphysik}

\usepackage[colorlinks=true,linkcolor=black]{hyperref}

\begin{document}
\maketitle
\titlelabel{Aufgabe \blatt{}.\thesection{} }

\section{Zyklotron}
\paragraph{Frequenz}
$$ ν_{Zyk} = \frac{1}{2π} \frac{q}{m} B = 22.11\;\mathrm{MHz} $$
Die Hochfrequenz $ν_{HF}$ liegt im Optimalfall etwas niedriger als die Zyklotronfrequenz $ν_{Zyk}$.

\paragraph{Maximalradius}
$$ r_{\mathrm{max}} = \frac{p}{qB} =  \frac{\sqrt{2m_p E_{kin}}}{qB} = 0.254\;\mathrm{m} $$

\paragraph{Geschwindigkeit}
$$ β_p = \sqrt{\frac{2E_{kin}}{m_p c^2}} = \frac{35\,288\,072\;\mathrm{\frac{m}{s}}}{c} = 0.1177 {\;}\hat{=}{\;} 11.77\;\% $$

\paragraph{Problematik}
Die relativistisch korrekte Zyklotronfrequenz ist:
$$ ω_{Zyk} = \frac{qB}{γm} = \mathrm{const.} $$
In einer nichtrelativistischen Näherung kann $γ=1$ angenommen werden, woraus ein konstantes Magnetfeld folgt.

Bei relativistischen Geschwindigkeiten muss sich demnach das Magnetfeld ändern, damit die Zyklotronfrequenz konstant bleibt.

\section{Wideroe Beschleuniger}
$$ \frac{E_{kin}}{U} = \frac{6\;\mathrm{MeV}}{0.1\;\mathrm{MeV}} = 60\;\mathrm{Beschleunigungsstrecken} $$

$$ l_n = \frac{1}{2} β λ = \frac{1}{2} \sqrt{\frac{2E_{kin}}{mc^2}} \frac{c}{f} = 3.55\;\mathrm{cm} $$


\end{document}