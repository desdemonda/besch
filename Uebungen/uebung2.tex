\documentclass[12pt,DIV=15,a4paper,twoside,draft=false]{scrartcl}
\usepackage[utf8]{inputenc}
\usepackage[ngerman]{babel}
\usepackage{makeidx}
\usepackage{graphicx}
%\usepackage{dsfont}
\usepackage{ulem}
\usepackage{amsthm}
\usepackage{amsmath}
\usepackage{wrapfig}
\usepackage{pdfpages}
\usepackage{uniinput}
\usepackage{float}
\usepackage{titlesec}

% Floats
\floatstyle{ruled}
\newfloat{bemer}{H}{lomark}
\floatname{bemer}{Bemerkung}

% Formatiert die Kopfzeilen und Fusszeilen
\usepackage{fancyhdr}
\pagestyle{fancy}
\setlength{\headheight}{25.2pt} % Damit alles in den Header passt.
\fancyhead[ER,OL]{\begin{minipage}{\textwidth}\begin{flushright}
\leftmark \\ \today
\end{flushright}
\end{minipage}}
\fancyhead[EL,OR]{\begin{minipage}{\textwidth}Abgabe von Jonathan Pieper \\ Übung zur Einführung in die Beschleunigerphysik
\end{minipage}}
\fancyfoot[EL,OR]{}
%\setlength{\headheight}{25.2pt} % Damit alles in den Header passt.
%\fancyhead[R]{\begin{minipage}{\textwidth}
%\leftmark \\ \today
%\end{minipage}}
%\fancyhead[L]{\begin{minipage}{\textwidth}Prof. Ratzinger\\ \rightmark \end{minipage}}

\newcommand{\antikommut}[2]{\{\hat{#1}, \hat{#2}\}}
\newcommand{\blatt}{2}

% Titelseite
\author{Jonathan Pieper}
\title{Abgabe Übung \blatt{}}
\subtitle{Übung zur Einführung in die Beschleunigerphysik}

\usepackage[colorlinks=true,linkcolor=black]{hyperref}

\begin{document}
\maketitle
\titlelabel{Aufgabe \blatt{}.\thesection{} }

\section{Relativistische Kinematik}
\begin{align*}
p &= \sqrt{2mT} &
E &= T + mc^2
\end{align*}
\paragraph{Fixed Target}
\begin{align*}
p_1 &= \sqrt{2 m_e 27.5\;\mathrm{GeV}} = 
8.959\cdot 10^{-20}\;\mathrm{Ns} = 168\;\mathrm{MeV/c} & 
p_2 &= 0\\
E_1 &= (27.5\;\mathrm{GeV} + m_e c^2) & 
E_2 &= (0 + m_p c^2)
\end{align*}
\begin{align*}
s^2 = m_{\mathrm{inv}}^2 &=
 (E_1 + E_2)^2 - (p_1c + p_2c)^2\\
  &=  ((27.5\;\mathrm{GeV} + m_e c^2) + (0+ m_p c^2))^2 - 2 m_e \cdot 27.5\;\mathrm{GeV} \cdot c^2\\
&= 8.087\cdot 10^{20}\;\mathrm{eV}^2 \\
⇒ \sqrt{s^2} &= 28.44\;\mathrm{GeV}
\end{align*}
\paragraph{Collider}
\begin{align*}
p_1 &= \sqrt{2 m_e 27.5\;\mathrm{GeV}} = 168\;\mathrm{MeV/c}& 
p_2 &= \sqrt{2 m_p 920\;\mathrm{GeV}} = 2.22\cdot 10^{-17}\;\mathrm{Ns} = 42\;\mathrm{GeV/c}\\
E_1 &= (27.5\;\mathrm{GeV} + m_e c^2) & 
E_2 &= (920\;\mathrm{GeV} + m_p c^2)
\end{align*}
\begin{align*}
s^2 = m_{\mathrm{inv}}^2 &=
 ((27.5\;\mathrm{GeV} + m_e c^2) + (920\;\mathrm{GeV} + m_p c^2))^2 - 
(168\;\mathrm{MeV}\cdot c + 42\;\mathrm{GeV}\cdot c)^2\\
&= 8.978\cdot 10^{23}\;\mathrm{eV}^2\\
⇒ \sqrt{s^2} &= 947.5 \;\mathrm{GeV}
\end{align*}

\section{Tandemgenerator}
$$ E = (e+q) U $$
\begin{description}
\item[$\mathrm{Ar^{3+}}$] $ E = 4 e U = 4 \;\mathrm{MeV}$
\item[$\mathrm{Ar^{4+}}$] $ E = 5 e U = 5 \;\mathrm{MeV}$
\item[$\mathrm{Ar^{5+}}$] $ E = 6 e U = 6 \;\mathrm{MeV}$
\end{description}
\end{document}