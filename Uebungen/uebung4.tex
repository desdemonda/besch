\documentclass[12pt,DIV=15,a4paper,twoside,draft=false]{scrartcl}
\usepackage[utf8]{inputenc}
\usepackage[ngerman]{babel}
\usepackage{makeidx}
\usepackage{graphicx}
%\usepackage{dsfont}
\usepackage{ulem}
\usepackage{amsthm}
\usepackage{amsmath}
\usepackage{wrapfig}
\usepackage{pdfpages}
\usepackage{uniinput}
\usepackage{float}
\usepackage{titlesec}
\usepackage{dsfont}

% Floats
\floatstyle{ruled}
\newfloat{bemer}{H}{lomark}
\floatname{bemer}{Bemerkung}

% Formatiert die Kopfzeilen und Fusszeilen
\usepackage{fancyhdr}
\pagestyle{fancy}
\setlength{\headheight}{25.2pt} % Damit alles in den Header passt.
\fancyhead[ER,OL]{\begin{minipage}{\textwidth}\begin{flushright}
\leftmark \\ \today
\end{flushright}
\end{minipage}}
\fancyhead[EL,OR]{\begin{minipage}{\textwidth}Abgabe von Jonathan Pieper \\ Übung zur Einführung in die Beschleunigerphysik
\end{minipage}}
\fancyfoot[EL,OR]{}
%\setlength{\headheight}{25.2pt} % Damit alles in den Header passt.
%\fancyhead[R]{\begin{minipage}{\textwidth}
%\leftmark \\ \today
%\end{minipage}}
%\fancyhead[L]{\begin{minipage}{\textwidth}Prof. Ratzinger\\ \rightmark \end{minipage}}

\newcommand{\Mv}[1]{\vec{#1}}
\newcommand{\blatt}{4}

% Titelseite
\author{Jonathan Pieper\and Dominik Kaufhold\and Kai Harry Schmidt}
\title{Abgabe Übung \blatt{}}
\subtitle{Übung zur Einführung in die Beschleunigerphysik}

\usepackage[colorlinks=true,linkcolor=black]{hyperref}

\begin{document}
\maketitle
\titlelabel{Aufgabe \blatt{}.\thesection{} }

\section{Betatron}
Das Prinzip des Betatrons basiert auf dem Faraday‘schen Induktionsgesetz:
\begin{align*}
U_{ind} &= \oint \Mv{E} d\Mv{s} = - \frac{d}{dt} ∫ \Mv{B} d\Mv{A} \\
 Φ &= ∫ \Mv{B} d\Mv{A} = πr^2 \bar{B}_z \\
 |\Mv{E}| &=  \frac{|U_{ind}|}{2πr} = \frac{πr^2}{2πr} \dot{\bar{B}}_z = \frac{1}{2} \dot{\bar{B}}_z r\\
 \dot{p} &=  q |\Mv{E}| = \frac{1}{2} q  \dot{\bar{B}}_z r = q \dot{B}_Z r\\
 ⇒ \dot{B}_Z &= \frac{1}{2} \dot{\bar{B}}_z
\end{align*}

$$ p_{max} = q B_Z^{max} r = q \cdot 1T \cdot 0.5\;\mathrm{m} = \frac{1}{2} e $$
%Der Strahlungsverlust durch Synchrotronstrahlung ist proportional zu $γ^4/r$.
%Daraus resultiert eine obere Grenze der Elektronenenergie bei ca.
%\begin{align*}
%E_{el} ≤ 300 - 500\;\mathrm{MeV} && ⇒ && {γ} ≤ 580 - 980
%\end{align*}


\section{Mikrotron}
$$ t= \frac{2πr}{v} = \frac{2πγm}{qB} = \frac{2πE}{qBc^2} $$

Im homogenen Magnetfeld ist die Umlaufzeit proportional zur Gesamtenergie. 
$$ ω = \frac{n c^2 B}{ΔE} $$
\end{document}
